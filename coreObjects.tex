\documentclass[11pt]{article}

\usepackage{float}
\title{XYO Network Encodings: Core Objects}
\author{0x02}
\date{August, 2018}
\begin{document}
\maketitle

\section{Bound Witness (0x01)}
A bound witness in the XYO Network.

\begin{center}
\begin{tabular}{ |l|l|l|l| } 
\hline
\textbf{Name} & \textbf{Type} & \textbf{Description}\\
\hline
Total Size & Unsigned Int & Size of Entire Bound Witness\\  
Public Keys & Array(0x02)$<KeySet>$ & A keyset for every party in the bound witness. \\  
Payloads & Array(0x03)$<Payload>$  & A payload for every party in the bound witness.\\  
Signatures & Array(0x02)$<SignatureSet>$ & A SignatureSet for every party in the bound witness.\\  
 
\hline
\end{tabular}
\end{center}

\section{KeySet (0x02)}
A set of public keys for a single party.
\begin{center}
\begin{tabular}{ |l|l|l|l| } 
\hline
\textbf{Name} & \textbf{Type} & \textbf{Description}\\
\hline
Keys & Array(0x05) & A multi-typed array of public keys.\\  
 
\hline
\end{tabular}
\end{center}

\section{SignatureSet (0x03)}
A set of signatures for a single party.
\begin{center}
\begin{tabular}{ |l|l|l|l| } 
\hline
\textbf{Name} & \textbf{Type} & \textbf{Description}\\
\hline
Signatures & Array(0x05) & A untyped array of signatures.\\  
 
\hline
\end{tabular}
\end{center}

\section{Payload (0x04)}
A payload contains a set of unsigned and signed Heuristics.
\begin{center}
\begin{tabular}{ |l|l|l|l| } 
\hline
\textbf{Name} & \textbf{Type} & \textbf{Description}\\
\hline
Total Size & Unsigned Int & Total size of Payload \\ 
Signed Heuristics & Array(0x06) & An multi-typed array of heuristics\\  
Unsigned Heuristics & Array(0x06) & An multi-typed array of heuristics\\  

 
\hline
\end{tabular}
\end{center}
\subsubsection*{Index (0x05)}
The index in an origin block.
\begin{center}
\begin{tabular}{ |l|l|l|l| } 
\hline
\textbf{Name} & \textbf{Type} & \textbf{Description}\\
\hline
Index & Unsigned integer & the index in the origin chain\\  
\hline
\end{tabular}
\end{center}
\subsubsection*{Previous Hash (0x06)}
The previous hash in an origin chain.
\begin{center}
\begin{tabular}{ |l|l|l|l| } 
\hline
\textbf{Name} & \textbf{Type} & \textbf{Description}\\
\hline
Hash Type Major & Byte & Hash Type \\ 
Hash Type Minor & Byte & Hash Type\\ 
Hash & Inferred above & the previous hash\\  
\hline
\end{tabular}
\end{center}
\subsubsection*{Next Public Key (0x07)}
The next public key to use in an origin chain.
\begin{center}
\begin{tabular}{ |l|l|l|l| } 
\hline
\textbf{Name} & \textbf{Type} & \textbf{Description}\\
\hline
Public Key Type Major & Byte & Public Key Type \\ 
Public Key Type Minor & Byte & Public Key Type\\ 
Next Public Key & Inferred above & The next public key in an origin chain.\\  
\hline
\end{tabular}
\end{center}

\end{document}

\end{document}

	