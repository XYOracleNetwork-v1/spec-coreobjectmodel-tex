\documentclass[11pt]{article}

\usepackage{float}
\title{XYO Network Encoding Scheme}
\date{August, 2018}
\begin{document}
\maketitle


\section{Overview}
Everything that is encoded is a superset of an \textit{Object}. An object is defined by having an object header. There are two types of objects, base objects and array objects.

\subsection{Base Objects}
Everything is a base object. A base object is defined by having an object header (see section 2).

\subsection{Array Objects}
An array object is a superset of a base object, but can have child objects. 

\section{Object Header}
A header to be placed before every object.

\subsection{Size Identifier Size (2 bits)}
2 bits placed at the beginning of the header to indicate how to read the size of the payload.

\subsubsection{00b and 01b}
This indicates to read the major (the next 6 bits) to use as the size of entire payload. 

\subsubsection{10b}
This indicates to read 2 bytes after the minor to obtain the size. 


\subsubsection{11b}
This indicates to read 4 bytes after the minor to obtain the size. 


\subsection{Major (6 bits)}
Used as an ID for the value/payload. If the Size Identifier Size is 00b or 01b then the major must be the size of the value/payload.

\subsection{Minor (1 byte)}
Used as a secondary ID for the value/payload.

\subsection{Size (2 or 4 bytes)}
Used to indicate the size (in bytes) of the value/payload. This is only included in the header if the Size Identifier Size is 10b or 11b. The size is always unsigned and includes itself.

\begin{center}
\begin{tabular}{ |l|l| } 
\hline
\textbf{Size Identifier Size} & \textbf{Value}\\
\hline
00b & None \\  
01b & None \\  
10b & 2 Bytes (Short) \\  
11b & 4 Bytes (Int) \\
\hline
\end{tabular}
\end{center}

\subsection{Examples}

\begin{center}
\begin{tabular}{ |l|l|l|l| } 
\hline
\textbf{Name} & \textbf{Value} & \textbf{Description}\\
\hline
Size identifier size & 00b & A 2 bit description of the size.\\  
Major & 000001b & The major of the item.\\ 
Minor & 0x1a & The minor of the item.\\ 
Value & 0x33 & The value is 1 byte because of the major.\\ 
\hline
\end{tabular}
\end{center}

\begin{center}
\begin{tabular}{ |l|l|l|l| } 
\hline
\textbf{Name} & \textbf{Value} & \textbf{Description}\\
\hline
Size identifier size & 01b & A 2 bit description of the size.\\  
Major & 000001b & The major of the item.\\ 
Minor & 0x2a & The minor of the item.\\ 
Value & 0x23 & The value is 1 byte because of the major.\\ 
\hline
\end{tabular}
\end{center}

\begin{center}
\begin{tabular}{ |l|l|l|l| } 
\hline
\textbf{Name} & \textbf{Value} & \textbf{Description}\\
\hline
Size identifier size & 10b & A 2 bit description of the size.\\  
Major & 000011b & The major of the item.\\ 
Minor & 0x2b & The minor of the item.\\ 
Size & 0x00,0x05 & The minor of the item.\\ 
Value & 0x01,0x02,0x03 & The value is 1 byte because of the major.\\ 
\hline
\end{tabular}
\end{center}

\begin{center}
\begin{tabular}{ |l|l|l|l| } 
\hline
\textbf{Name} & \textbf{Value} & \textbf{Description}\\
\hline
Size identifier size & 11b & A 2 bit description of the size.\\  
Major & 000011b & The major of the item.\\ 
Minor & 0x2b & The minor of the item.\\ 
Size & 0x00,0x00,0x00,0x04 & The minor of the item.\\ 
Value & 0x13,0x37 & The value is 1 byte because of the major.\\ 
\hline
\end{tabular}
\end{center}

\end{document}
